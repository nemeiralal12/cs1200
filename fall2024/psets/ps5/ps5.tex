\documentclass[11pt]{article}
\usepackage{classTools}
\usepackage{tikz}
\usepackage{chessboard}
\usepackage{xcolor}

\begin{document}

% To include a problems set header, use the psHeader command
\psHeader{5}{Wed Oct. 23, 2024 (11:59pm)}

\textbf{Your name: Nemeira Lal}

\textbf{Collaborators: Faseeh, Raphael, Vennela, Bautista}

\textbf{No. of late days used on previous psets: 6}

\textbf{No. of late days used after including this pset: 6}

\medskip \noindent
\textit{Remember to mark your pages on Gradescope properly, or points will be taken off. Additionally, if your handwriting isn't particularly neat please submit written proofs, which are much easier to grade for the TFs.}

\begin{enumerate}
    \item (Solving Games) Consider playing a solo game on an $n\times n$ chessboard.  You have one piece, a chess knight, which starts in the lower-left corner, and your goal is to reach any of the other three corners in as few moves as possible.  Like a usual chess knight, in one move, you can move to any position that is two squares away in a horizontal direction and one square away in a vertical direction, or two squares away in a vertical direction and one square away in a horizontal direction.  There is a catch, however: some squares have visible landmines, so you cannot move to them (since you do not want set off an explosion). \

    \begin{enumerate}
    \item Give an algorithm that achieves the above goal in time $O(n^2)$ by reduction to either the ShortestWalks problem or the SingleSourceShortestPaths problem. The algorithm should output $\bot$ if no sequence of moves can take you to any of the other three corners when started in the lower-left corner. (Note: if reducing to SingleSourceShortestPaths, we haven't defined abstractly what it means to reduce a computational problem to a data structure problem, but you may construct and use a SingleSourceShortestPaths data structure on an appropriately constructed graph.)
    \\\\ \textbf{Data Structure:} Firstly, let's create our data structure. We can here envision a graph structure, with the vertices being all possible squares that we can land on that don't have a mine.
    \\ So, in $G = (V, E)$, the $V$ will be the set of all non-mine squares on the chess board.
    \\ As for the $E$, those will connect all the vertices, or squares that the horse can go between. For example, there will be a edge between the vertices signifying the boxes $(1, a)$ and $(2, c)$. 
    \\\\ \textbf{Algorithm:} Based on this graph G, we can then find the shortest distance between any of the 4 corners by looking for $dist_G(s, t)$. where $s$ is the starting box $(1, a)$, and $t$ is any one of the corners. 
    \\ We can use the \textbf{Breadth First Search} Algorithm here to look for all the possible paths between the initial starting point and all other vertices, stopping the moment we hit any of 3 corners. If there are no short paths to the corners, we return none.
    \\\\ \textbf{Time complexity:} We have $n^2$ different squares, and to create a graph with those number of squares we would need time of $O(n^2)$. Then, to go through each vertex once with BFS, we would at most also need $8\cdot O(n^2) = O(n^2)$ time. Hence, the total runtime would be: $O(n^2)$.
    \\
    \item Carry out your algorithm on the $5\times 5$ board shown below, listing both the frontier vertices and the predecessor relationships at each stage of BFS (the red square symbols are landmines).
    
    \begin{figure} [H]
    \centering
    \begin{tikzpicture}
        \definecolor{white}{rgb}{1,1,1}
        \definecolor{black}{rgb}{0.5,0.5,0.5}
        \definecolor{darkred}{rgb}{0.6, 0, 0}

        \def\landmines{(2.5,1.5), (1.5,5.5), (5.5,1.5), (1.5,3.5), (3.5,1.5),(3.5,3.5),(4.5,4.5),(4.5,5.5),(2.5,4.5), (4.5,3.5)}
        
        % Loop to create the 5x5 grid
        \foreach \x in {1,...,5} {
            \foreach \y in {1,...,5} {
                % Alternate black and white squares
                \pgfmathparse{mod(\x+\y,2)}
                \ifnum\pgfmathresult=0
                    \fill[white] (\x, \y) rectangle ++(1,1);
                \else
                    \fill[black] (\x, \y) rectangle ++(1,1);
                \fi
            }
        }
        % X-axis labels
        \foreach \x/\letter in {1/a, 2/b, 3/c, 4/d, 5/e} {
            \node at (\x+0.5, 0.5) {\letter}; 
        }
    
        % Y-axis labels
        \foreach \y in {1, 2, 3, 4, 5} {
            \node at (0.5, \y+0.5) {\y};  
        }
    
        % Knight symbol
        \node at (1.5, 1.5) {\knight};

        %Adding landmines
        \foreach \position in \landmines {
            \draw[fill=red, opacity=0.5] \position ++(-0.3,-0.3) rectangle ++(0.6,0.6); 
            \node at \position {\textcolor{darkred}{\Huge$\circ$}};  
        }
    
        % Grid lines
        \draw[step=1cm,black,thick] (1,1) grid (6,6);
        
    \end{tikzpicture}
    \end{figure}
    Let's first create our data structure. 
    \\ $E = {(1, a), (2, a), (4, a), (2, b), (3, b), (5, b), (2, c), (4, c), (5, c), (1, d), (2, d), (2, e), (3, e), (4, e), (5, e)}$
    \\ $V = ((1, a), (2, c)), ((2, b), (4, c)), .....$ where each vertex i connects to the vertices the horse could land on if it began at i.
    \\\\ To start the algorithm, let's create a Frontier array which we will keep updating as we iterate through the board. We will terminate the algorithm once one of the 3 corners enters the Frontier array. We will also have an array S that will keep track of all the squares explored.
    \\\\ \textbf{Step 1:} We begin at the home square in the array S, $(1, a)$. The Frontier array holds all the squares that the horse can move into from $(1, a)$, or all the vertices in $V$ that connect to the initial vertex. That would be: $(2, c)$ and $(3, b)$.
    \\ \textbf{Step 2: } Now, using the Frontier array from the last step, we will add all those vertices to the array S to mark them as explored. We will then empty out the Frontier array and use S to create a new one based on all the vertices in S. So, our new frontier array will have the vertices $(3, e), (5, c), (2, d)$.
    \\ \textbf{Step 3: } We will repeat the same step again, and our new Frontier array will now have the squares: $(4, c), (4, e)$. We can potentially have other points like $(2, c)$, but we have already explored them and they are in S so we don't want to explore them again.
    \\ \textbf{Step 4: } This iteration of our Frontier array would be: $(5, e), (5, c), (2, d)$. As you can see, this contains one of the corners $(5, e)$, so we can terminate our program right here and return.
    \\ \textbf{Completion:} We have successfully reached another corner using the BFS by translating this board to a graph.

    \end{enumerate}


 \item (Maximal Independent Sets) \label{prob:maximalIS} Let $G=(V,E)$ be a graph.  A set $S\subseteq V$ is a {\em maximal} independent set if we cannot add any vertices to $S$ while it remains an independent set.  That is, for every vertex $v\in V\setminus S$, $S\cup\{v\}$ is not an independent set.
    \begin{enumerate}
        \item Show that given a graph $G$, a maximal independent set can be found in time $O(n+m)$.  Note that this is in sharp contrast to {\em maximum-size} independent sets, for which we do not know any subexponential-time algorithms. (Hint: be greedy.)
        \\\\ We can use a greedy algorithm to find the maximal independent set in $O(n + m)$ time.
        \\\\ \textbf{Algorithm: }We will begin by having two different structures, one being an array of the Maximal set we want, and the other being an array of all the visited vertices, so we can keep track of every node we have either visited or is adjacent to one of our vertices in the independent set.
        \\ We begin with the first vertex i, and then we can add this vertex to both the Maximal Set and also the visited set. Then, we mark of i's adjacent vertices to be visited. Because i is in the maximal set, none of its adjacent vertices can be in the set. So, we mark them visited so we don't visit them again. Hence, our algorithm them takes us to the next set of vertices that are not adjacent to i, and haven't been adjacent yet. We follow the same steps by first adding them to the Maximal set, and then adding them and all their adjacent vertices into the visited set. We keep doing this until we get our maximal set.
        \\\\ \textbf{Proof of Correctness: }Because we mark each of the adjacent nodes of any vertex we add to the Maximal set as visited, we make sure our algorithm never traverses through them again and adds them to the Maximal set as well. That makes sure that the set remains independent and has no adjacent vertices. We can prove that it is indeed maximal because after the algorithm ends, there are no remaining vertices that are either not visited themselves or are adjacent to a vertex in the maximal set already. So, adding to S after the algorithm halts will mean the property of independence does not hold anymore.
        \\\\ \textbf{Runtime: } Each vertex and edge is visited once, so hence the runtime is $O(n + m)$.
        \\
        \item Show that if $G$ is 3-colorable, then it has a 3-coloring $f$ in which the set of vertices of color 2 (i.e. $f^{-1}(2)$) is a maximal independent set.
        \\\\ Let's first define what it means for $G$ to be 3-colorable! It means that there is a function $f$ such that for each $(u, v) \in E$, $f(u) \neq f(v)$. Essentially means that no two adjacent vertices are the same color.
        \\\\ Let's now define the vertices of color 2: $f^{-1}(2)$. For this set to be maximal, it would mean that adding any more vertex into the set $f^{-1}(2)$, the set would fail to remain independent- or we are adding an adjacent node to one of the existing nodes in the set. 
        \\\\ \textbf{Direct Proof:} Assume if $G$ is 3-colorable, and we have our set $f^{-1}(2)$. If $f^{-1}(2)$ is not maximal, then there exists a $v \in V$ that is independent to all vertices in $f^{-1}(2)$. So, we can color it with 2. We can keep doing so until there are no more vertices left that are independent to $f^{-1}(2)$, and we would be left with a maximal independent set. So, if G is 3-colorable, then there exists a f such that $f^{-1}(2)$ is maximally independent. \\

        \item It is known that every graph $G$ has at most $3^{n/3}$ maximal independent sets, and there is an algorithm (the Bron-Kerbosch algorithm) that enumerates all of the maximal independent sets in time $O(3^{n/3}).$  Use this fact to conclude that 3-coloring can be solved in time $O((n+m)\cdot 3^{n/3}) = O(1.44^n),$ improving the runtime of $O(1.89^n)$ from SRE4.\\
        \\ First of all, looking at the last problem, we see that if $G$ is a 3-colorable graph, then there exists a 3-coloring where the sets of at least one color is maximally independent. 
        \\\\ So, using Bron-Kerbosch's algorithm, we can find all the maximally independent sets in $O(3^{n/3} = 1.44^n)$ time. Then, for each of these independent sets, we can try to 2-color the vertices not in this set. That will take $O(n+m)$ time.
        \\\\ \textbf{Runtime:} $O(1.44^n) \cdot (m+n)$ = $O(1.44^n)$ because the latter is dominates, and which is better than $O(1.89^n)$
    \end{enumerate}
 
 \item (Exponential-Time Coloring) 
  In the \href{https://github.com/Harvard-CS-120/cs120/tree/main/fall2022/psets}{Github repository} for PS5, we have given you basic data structures for graphs (in adjacency list representation) and colorings, an implementation of the Exhaustive-Search $k$-Coloring algorithm, 
  an implementation of the Bron-Kerbosch algorithm, and a variety of test cases (graphs) for coloring algorithms. 

  \begin{enumerate}
      \item Implement the $O(n+m)$-time algorithm for 2-coloring that we covered in class in the function \texttt{bfs\_2\_coloring}, verifying its correctness by running \texttt{python3 -m ps5\_tests 2}.
        Your implementation of BFS should follow the presentation and notation that we used in class (with the loop over distance $d$ and the sets $F$ and $S$), which may be different than presentation of BFS in other sources (online or in the optional textbooks). 
        
      \item Implement the $O((n+m)\cdot 3^{n/3})$-time algorithm for 3-coloring (MaximalIS + BFS) from Problem~\ref{prob:maximalIS} above in the function \texttt{iset\_bfs\_3\_coloring}, also verifying its correctness by running \texttt{python3 -m ps5\_tests 3}. \label{part:TbT}
    
    \item Compare the efficiency of Exhaustive-Search 3-coloring and the $O((n+m)\cdot 3^{n/3})$-time algorithm. Specifically, identify and write down the largest instance size $n$ each algorithm is able to solve (within a time limit you specify, e.g. 1 second) and the smallest instance size $n$ each algorithm is unable to solve (again within that same time limit). 
    
    In addition to these numeric values, please provide a brief explanation of why these results make sense, based on your knowledge of both the algorithms' runtime and how each algorithm goes about finding a coloring. For this part, there is no need to go through every combination of parameters; feel free to give just the largest and smallest instances each algorithm can solve and speak generally as to why one algorithm performs better than the other. More instructions can be found in \texttt{ps5\_experiments}.
    \\\\ For any number of rings, the smallest n solved for both the two algorithms is 4. However, as for the \textbf{maximum}, we have something that looks like:
    \\\\ \textbf{Exhaustive Coloring in Ring Graphs:} n = 36
    \\ \textbf{ISET BFS Coloring in Ring Graphs:} n = 1000
    \\ \textbf{Exhaustive Coloring in Cluster Connections:} n = 36
    \\ \textbf{ISET BFS Coloring in Cluster Connections:} n = 54
    \\\\ The results obviously show that the Exhaustive Coloring in the Ring Graphs perform better than the ISET BFS Coloring for both the Ring Graphs and the Cluster Connections. The simple explanation is that with the Exhaustive Coloring algorithm, we consistently go through every possible coloring, which is an immense amount of computation and memory used as n increases. on the other hand, the ISET algorithm strategy uses the structure of the graph to explore fewer color configurations and avoid the combinatorial explosion and have extreme number of different colorings to explore.
    
  \end{enumerate}
  
  

\item (Reflection): Take one homework problem you have worked on this semester that you struggled to understand and solve, and explain how the struggle itself was valuable.  In the context of this question, describe the struggle and how you overcame the struggle. You might also discuss whether struggling built aspects of character in you (e.g. endurance, self-confidence, competence to solve new problems) and how these virtues might benefit you in later ventures. 

 \textit{Note: As with the previous psets, you may include your answer in your PDF submission, but the answer should ultimately go into a separate Gradescope submission form.}

 \textit{Quick note on grading: Good responses are usually about a paragraph, with something like 7 or 8 sentences. Most importantly, please make sure your answer is specific to this class and your experiences in it. If your answer could have been edited lightly to apply to another class at Harvard, points will be taken off.}

\item Once you're done with this problem set, please fill out \href{https://forms.gle/pR1Mt4PmMhUfuW1G6}{this survey} so that we can gather students' thoughts on the problem set, and the class in general. It's not required, but we really appreciate all responses!
\end{enumerate}

\end{document}
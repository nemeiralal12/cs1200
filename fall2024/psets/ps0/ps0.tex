\documentclass[11pt]{article}
\usepackage{classTools}
\usepackage{graphicx}
\drafttrue

\begin{document}
\psHeader{0}{Wed 2024-09-11 (11:59PM)}
\textbf{Q1) a)} Code in Python file! The reason this is O(n) because the search makes us go through all of the n nodes.
\\\\
\textbf{Q1) b)}(proofs by induction) Let $t$ be a positive integer. Prove that every binary tree $\treeT$ of size at least $2t+1$ has a vertex $\btv$ such that the subtree rooted at $\btv$ has size at least $t$ and at most $2t$.  (Hint: use induction on the size $n$ of $\treeT$, starting with base case $n=2t+1$.  Remember that nodes in binary trees can have 2, 1, or 0 children.) \label{part:induction-tsize}
    \\\\
    \textbf{Claim:} Every binary tree T of size at least 2t+1 has a vertex v such that the subtree rooted at v has size at least t and at most 2t
    \\ \textbf{Base Case:} If size of tree $N = 2t + 1$, then there exists a v such that size of the subtree at $v \leq 2t$ and $v \geq t$
    \\ \textbf{Proving Base Case:} Proving by contradiction: Assume all vertices have sizes less than t or more than 2t. 
    \\ If both the children of the tree with the size 2t + 1 are less than t, then they will not add up to 2t. So, this tree can not exist as the children's size + 1 will not add up to the parent's size.
    \\ If either one of them is larger than 2t, then they will add up to a size larger than 2t, which is also not possible, because the size of the parent is 2t + 1, and the children need to add up to 2t exactly.
    \\ Edge Case: Only one child. Then, the size of the vertex has to be 2t, which fulfills the condition to begin with.
    \\ So, regardless of any of the cases, there will be a vertex in the base case that satisfies the claim.
    \\ \textbf{Strong Inductive Hypothesis: } Given $k \geq 2t + 1$, for every n such that $2t + 1 \leq n \leq k$, there exists a vertex v that satisfies are claim. 
    \\ \textbf{Inductive Step: } For a tree of size k + 1, there exists a vertex v such that $t\leq size \leq 2t$
    \\ \textbf{Proof: }
    \\ We can break down the larger tree into 2 different situations:
    \\ (1) It has only one child: In this case, the size of the child has to be k. Referring back to the inductive hypothesis, we know that there will be a vertex v in that subtree that will satisfy this condition.
    \\ (2) It has 2 children: This case can be further broken down into three situations.
    \\ (2a) Either of the children has a size larger than 2t + 1: In this case, we can again go back to the inductive hypothesis and deduce that there will be a v that satisfies the claim.
    \\ (2b) Both are smaller than t: In this case, they will never be able to add up to 2t + 1. So despite failing the claim, this situation will never exist.
    \\ (2c) None of them are larger than 2t + 1, but one of them is larger than t: In this case, we know that the vertex fits the claim instantly, so we know that there exists a v that satisfies the claim.
    \\\\
    
\textbf{Q1) c)} Code in Python file! O(h) because only check the depth, so don't need to traverse through every single node. Regardless of the path we go down, we hit only one child at each depth level, and so the total number of nodes we check is exactly h.
\\\\
\textbf{Q2) a)} For each of the following pairs of functions, determine whether $f=O(g)$ and whether $f=o(g)$.  Justify your answers. \\
    
    \textbf{i)} $f(n) = 3\log_2^3 n$, $g(n) = n^2+1$.\\

    To prove if $f = O(g)$, we can also prove that $f = o(g)$, because $f = o(g)$ is more restrictive than the former. To prove the little-o relationship, we can look at it in limit format.
        \\ If $f = o(g)$, then:
    \begin{equation*}
        \lim_{x\to\infty}{\frac{f(n)}{g(n)}} = 0
    \end{equation*}
    \\ Plugging in our functions:
    \begin{equation*}
         \lim_{x\to\infty}{\frac{3(\log_2(n))^3}{n^2+1}}
    \end{equation*}
    \\ Because they both go to $\infty$ we can use l'hopital's rule, and take the derivative of top and bottom to get:
    \begin{equation*}
        \lim_{x\to\infty}\frac{\frac{d}{dx}3(\log_2(n))^3}{\frac{d}{dx}n^2+1}
    \end{equation*}
    After repeating the derivative step until we eliminate the indeterminate form, we get:
    \begin{equation*}
        \lim_{x\to\infty}\frac{1}{n^2} = 0
    \end{equation*}
    Because we just proved that the limit tends to 0, we can conclude that $f = o(g)$. As we had said before, this also means that $f = O(g)$.
    \\\\
    \textbf{ii)} $f(n) = 4n^3$, $g(n)= \left| \{S \subseteq [n] : |S|\leq 3\}\right|,$ where $[n]=\{0,1,2,\ldots,n-1\}$.  
    \\\\ 
    Let's put this into limit format again to check for the little-o relationship. 
    
    \begin{equation*}
        lim_{x\to\infty}\frac{4n^3}{ \left| \{S \subseteq [n] : |S|\leq 3\}\right|}
    \end{equation*} \\
    To understand what the denominator, or $g(n)$ means, we can see it is the cardinality, or the size of all the sets of the larger set [n] such that the size of the subsets are less or equal to 3.
    \\ That means we are simply adding:
    \begin{equation*}
        {n \choose{1}} + {n\choose{2}} + {n \choose{3}} + 1
         = n + \frac{(n)(n-1)}{2} + \frac{(n)(n-1)(n-2)}{6} + 1
    \end{equation*}
    When we put this expression into the limit in the denominator, it would approximate to the $n^3$ because the highest order of the expression is of the term $n^3$
    \\ Then, the limit simplifies to:
    \begin{equation*}
        lim_{x\to\infty}\frac{n^3}{n^3} = constant \neq 0
    \end{equation*}
    \\ As we can see, the limit does NOT equal 0. So, we are proving that the relationship between $f(n)$ and $g(n)$ is not that $f = o(g)$
    \\ \\ As for the Big-O relationship, we see that as n gets infinitely large, both functions go to an $n^3$ relationship. This is the case, so that means that there has to exist a $c > 0$ such that $c * g(n) \geq f(n)$.
    \\\\
    \textbf{iii)} $f(n) = 5^n$, $g(n)=n!$.
    \\\\
    First, let's write this in the form of a limit:
\begin{equation*}
    lim_{x\to\infty}\frac{5^n}{n!} = \frac{1*2*3*...*n}{5*5*5*....*5}
\end{equation*}
\\As you can see, as n gets larger, the denominator gets increasingly multiplied by terms even larger, unlike the numerator which continuously gets multiplied by 5.
\\ So, 
\begin{equation*}
      lim_{x\to\infty}\frac{5^n}{n!} = 0
\end{equation*}
    In other terms, as n tends to infinity, the function $g(n)$ increases by a factor of n, versus $f(n)$ increases by a factor of 5. This shows very clearly that $g(n)$ grows immensely faster than $f(n)$. This, and the limit being equal to 0, means we can confirm that both $f = o(g)$ and $f = O(g)$ are true.
    \\\\
\textbf{Q2) b) }Prove or disprove: For all functions $f,g : \N\rightarrow \R^+,$ $f=O(g) \Rightarrow g\neq o(f)$.
\\\\
Let's set $f = O(g)$. This would imply that there exists a $c$ such that $f(n) \leq c*g(n)$ as n gets larger.
\\
Rewriting this, we get:
\begin{equation*}
    \frac{1}{c} \leq \frac{g(n)}{f(n)}
\end{equation*}
As $c > 0$, we know that $\frac{1}{c} > 0$ as well.
\\ So:
\begin{equation*}
     lim_{x\to\infty}\frac{g(n)}{f(n)} \geq \frac{1}{c} > 0
\end{equation*}
\\ So, $g \neq o(f)$.
\end{document}

\iffalse